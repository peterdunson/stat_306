% Options for packages loaded elsewhere
\PassOptionsToPackage{unicode}{hyperref}
\PassOptionsToPackage{hyphens}{url}
\documentclass[
]{article}
\usepackage{xcolor}
\usepackage[margin=1in]{geometry}
\usepackage{amsmath,amssymb}
\setcounter{secnumdepth}{-\maxdimen} % remove section numbering
\usepackage{iftex}
\ifPDFTeX
  \usepackage[T1]{fontenc}
  \usepackage[utf8]{inputenc}
  \usepackage{textcomp} % provide euro and other symbols
\else % if luatex or xetex
  \usepackage{unicode-math} % this also loads fontspec
  \defaultfontfeatures{Scale=MatchLowercase}
  \defaultfontfeatures[\rmfamily]{Ligatures=TeX,Scale=1}
\fi
\usepackage{lmodern}
\ifPDFTeX\else
  % xetex/luatex font selection
\fi
% Use upquote if available, for straight quotes in verbatim environments
\IfFileExists{upquote.sty}{\usepackage{upquote}}{}
\IfFileExists{microtype.sty}{% use microtype if available
  \usepackage[]{microtype}
  \UseMicrotypeSet[protrusion]{basicmath} % disable protrusion for tt fonts
}{}
\makeatletter
\@ifundefined{KOMAClassName}{% if non-KOMA class
  \IfFileExists{parskip.sty}{%
    \usepackage{parskip}
  }{% else
    \setlength{\parindent}{0pt}
    \setlength{\parskip}{6pt plus 2pt minus 1pt}}
}{% if KOMA class
  \KOMAoptions{parskip=half}}
\makeatother
\usepackage{graphicx}
\makeatletter
\newsavebox\pandoc@box
\newcommand*\pandocbounded[1]{% scales image to fit in text height/width
  \sbox\pandoc@box{#1}%
  \Gscale@div\@tempa{\textheight}{\dimexpr\ht\pandoc@box+\dp\pandoc@box\relax}%
  \Gscale@div\@tempb{\linewidth}{\wd\pandoc@box}%
  \ifdim\@tempb\p@<\@tempa\p@\let\@tempa\@tempb\fi% select the smaller of both
  \ifdim\@tempa\p@<\p@\scalebox{\@tempa}{\usebox\pandoc@box}%
  \else\usebox{\pandoc@box}%
  \fi%
}
% Set default figure placement to htbp
\def\fps@figure{htbp}
\makeatother
\setlength{\emergencystretch}{3em} % prevent overfull lines
\providecommand{\tightlist}{%
  \setlength{\itemsep}{0pt}\setlength{\parskip}{0pt}}
\usepackage{bookmark}
\IfFileExists{xurl.sty}{\usepackage{xurl}}{} % add URL line breaks if available
\urlstyle{same}
\hypersetup{
  pdftitle={306\_takehome\_exam1\_peterdunson},
  pdfauthor={Peter},
  hidelinks,
  pdfcreator={LaTeX via pandoc}}

\title{306\_takehome\_exam1\_peterdunson}
\author{Peter}
\date{2026-02-21}

\begin{document}
\maketitle

STAT 306 --- Bayesian Statistics

Take-Home Exam 1 Spring 2026

Your written solutions to this exam are due at classtime - 10:10 am
Wednesday February 25, 2026.

For this exam, you may use your text, class notes, your old homeworks,
my written homework solutions and any course handouts (basically, you
may use any written materials specific to this class.) As for
technology, you may use your own hand calculator, R (including my R
scripts on Moodle), Python, Maple, Excel, Desmos, Wolfram, and other
comparable calculation software programs (please cite what you use).

However, - you MAY NOT consult with any other person, living or dead -
you MAY NOT search for solutions on the internet - you MAY NOT use
generative AI in any of its forms

In other words, the work you submit must originate from YOU!

Please sign the following pledge: The work I present here is completely
my own. In all phases of the completion of this exam, I NEITHER accepted
assistance from NOR gave assistance to any other person. In addition, I
DID NOT search the internet for any solutions and I DID NOT employ
generative AI.\\

Your Signature:
\_\_\_\_\_\_\_\_\_\_\_\_\_\_\_\_\_\_\_\_\_\_\_\_\_\_\_\_\_\_\_\_\_\_

\newpage

\subsection{Problem 1}\label{problem-1}

\begin{enumerate}
\def\labelenumi{\arabic{enumi}.}
\tightlist
\item
  On the kitchen counter, a cookie jar initially contained three Oreo
  cookies and five sugar cookies (almost white in color). When someone
  pulls out an Oreo cookie, there is always that characteristic residue
  of fine black crumbs left on the counter. Initially the counter is
  completely clean, but after the five-year-old son of the family
  reaches in and pulls out three cookies at random, one at a time, his
  mom notices the tell-tale scatter of black crumbs on the counter.
  Given this forensic evidence, find
\end{enumerate}

\textbf{Hypergeometric}

\subsubsection{{[}a{]} the conditional probability that the boy drew
exactly two Oreos in his sample of three cookies. Show your
calculations.
{[}12{]}}\label{a-the-conditional-probability-that-the-boy-drew-exactly-two-oreos-in-his-sample-of-three-cookies.-show-your-calculations.-12}

3 oreo cookies, 5 sugar cookies\\

At least 1 of 3 pulled is oreo.\\

Total number of ways to pull 3 cookies from 8 is
\(\binom{8}{3} = 56\).\\

\$\$ P(X=k): \newline

P(X=0) = \frac{\binom{3}{0} \cdot \binom{5}{3}}{56} = \frac{10}{56}
\newline P(X=1) = \frac{\binom{3}{1} \cdot \binom{5}{2}}{56} =
\frac{30}{56} \newline P(X=2) =
\frac{\binom{3}{2} \cdot \binom{5}{1}}{56} = \frac{15}{56} \newline
P(X=3) = \frac{\binom{3}{3} \cdot \binom{5}{0}}{56} = \frac{1}{56}
\newline

\$\$

\(P(X=2 \mid X \geq 1) = \frac{P(X=2, X \geq 1)}{P(X \geq 1)}\)

Given our pmf above, \(P(X=2, X \geq 1) = \frac{15}{56}\) and
\(P(X>=1) = \frac{46}{56}\), so:

\$\$ P(X=2 \mid X \geq 1) = \frac{15}{46} = 0.32608695652

\$\$

\subsubsection{{[}b{]} the conditional probability that the boy's third
cookie drawn was a sugar cookie. Show your calculations.
{[}12{]}}\label{b-the-conditional-probability-that-the-boys-third-cookie-drawn-was-a-sugar-cookie.-show-your-calculations.-12}

Total outcomes where \(X \geq 1 = 30 + 15 + 1 = 46\).\\

Let \(S_3\) be the event that the 3rd cookie is Sugar. We need
\(P(S_3 \mid X \geq 1)\).\\

Total number of Sugar cookies across all \(X \geq 1\) cases:\\

\(X=1\) (2 Sugar): \(30 \cdot \frac{2}{3} \cdot 3 = 60\) sugar cookies\\
\(X=2\) (1 Sugar): \(15 \cdot \frac{1}{3} \cdot 3 = 15\) sugar cookies\\
\(X=3\) (0 Sugar): \(0\) sugar cookies\\

Total sugar cookies in our reduced sample space =
\(60 + 15 + 0 = 75\).\\

Since each of the 3 positions (1st, 2nd, 3rd) is equally likely to hold
any of the cookies in the sample:\\

\$\$ P(S\_3 \mid X \geq 1) = P(S\_3 \mid X \geq 1) =
\frac{\sum S_i}{n \cdot (\text{Valid Outcomes})} \newline

P(S\_3 \mid X \geq 1) = \frac{75}{46 \cdot 3} = \frac{75}{138} \newline

P(S\_3 \mid X \geq 1) = \frac{25}{46} = 0.54347826087 \$\$

\subsubsection{\texorpdfstring{{[}c{]} the expectation of the number of
Oreos eaten by the boy. {[}10{]}\\
}{{[}c{]} the expectation of the number of Oreos eaten by the boy. {[}10{]} }}\label{c-the-expectation-of-the-number-of-oreos-eaten-by-the-boy.-10}

We need the conditional expectation \(E[X \mid X \geq 1]\).\\

\(P(X=k \mid X \geq 1) = \frac{P(X=k)}{P(X \geq 1)}\):\\

\(P(X=1 \mid X \geq 1) = \frac{30/56}{46/56} = \frac{30}{46}\)\\
\(P(X=2 \mid X \geq 1) = \frac{15/56}{46/56} = \frac{15}{46}\)\\
\(P(X=3 \mid X \geq 1) = \frac{1/56}{46/56} = \frac{1}{46}\)\\

So, the expectation is the weighted sum of these outcomes:\\

\$\$ E{[}X \mid X \geq 1{]} = \sum\_\{k=1\}\^{}\{3\} k \cdot P(X=k
\mid X \geq 1) \newline

E{[}X \mid X \geq 1{]} = \left(1 \cdot \frac{30}{46}\right) + \left(2
\cdot \frac{15}{46}\right) + \left(3 \cdot \frac{1}{46}\right) \newline

E{[}X \mid X \geq 1{]} = \frac{30 + 30 + 3}{46} = \frac{63}{46} \newline

E{[}X \mid X \geq 1{]} = 1.36956521739 \$\$

\newpage

\subsection{Problem 2}\label{problem-2}

\begin{enumerate}
\def\labelenumi{\arabic{enumi}.}
\setcounter{enumi}{1}
\tightlist
\item
  A player rolls a fair die. If it is not a six, then the player's turn
  is over and his score is the value showing on the die. However, if the
  die value is six, he gets to roll one more time as a bonus! The
  player's score X is then the sum of his two rolls. For example,
  consider the following die sequences, and the subsequent score for
  that player's turn.
\end{enumerate}

rolls : (6,1) score X = 7 rolls : (6,6) score X = 12 rolls : (4) score X
= 4 rolls : (3) score X = 3

\subsubsection{{[}a{]} Find the probability mass function for X (a table
will suffice). Please don't simplify probability fractions --- keep them
with the same denominator so they're easy to compare.
{[}10{]}}\label{a-find-the-probability-mass-function-for-x-a-table-will-suffice.-please-dont-simplify-probability-fractions-keep-them-with-the-same-denominator-so-theyre-easy-to-compare.-10}

\subsubsection{{[}b{]} Verify that your probabilities in {[}a{]} sum to
an appropriate value.
{[}8{]}}\label{b-verify-that-your-probabilities-in-a-sum-to-an-appropriate-value.-8}

\subsubsection{{[}c{]} Find the expectation and variance for X.
{[}16{]}}\label{c-find-the-expectation-and-variance-for-x.-16}

\subsubsection{{[}d{]} If we know that the player's score is at least 4,
then find the conditional probability that she got a bonus roll.
{[}8{]}}\label{d-if-we-know-that-the-players-score-is-at-least-4-then-find-the-conditional-probability-that-she-got-a-bonus-roll.-8}

\newpage

\subsection{Problem 3}\label{problem-3}

\begin{enumerate}
\def\labelenumi{\arabic{enumi}.}
\setcounter{enumi}{2}
\tightlist
\item
  Sometimes when it is desired to get survey data on a sensitive issue,
  for example whether or not students have ever cheated on an exam, the
  following method is employed. The question is ``Have you ever cheated
  on an exam?'', and the student answering the question flips a fair
  coin --- if the coin lands heads the student answers ``Yes''
  regardless of the truth, and if the coin lands tails, the student
  answers truthfully. Then a Yes answer is diluted by a plethora of
  false positives, giving security to a student that answers Yes
  genuinely. This survey sampling strategy is actually very useful in
  getting data on sensitive issues by backing out the false positives
  --- so clever! Suppose in a very large student population, a
  proportion p of the students have cheated on an exam in their past.
\end{enumerate}

\subsubsection{{[}a{]} If a student is chosen at random from this
population, and takes the survey by flipping the coin, and answering
accordingly, then what is the probability the student answers Yes?
{[}10{]}}\label{a-if-a-student-is-chosen-at-random-from-this-population-and-takes-the-survey-by-flipping-the-coin-and-answering-accordingly-then-what-is-the-probability-the-student-answers-yes-10}

\subsubsection{{[}b{]} Suppose five students from this population are my
advisees, and upon taking the coin-flip survey, all five answer Yes.
Given this information, what is the probability that at least one of my
advisees has cheated on an exam? Your answer will of course be a
function of
p.~{[}16{]}}\label{b-suppose-five-students-from-this-population-are-my-advisees-and-upon-taking-the-coin-flip-survey-all-five-answer-yes.-given-this-information-what-is-the-probability-that-at-least-one-of-my-advisees-has-cheated-on-an-exam-your-answer-will-of-course-be-a-function-of-p.-16}

\newpage

\subsection{Problem 4}\label{problem-4}

\begin{enumerate}
\def\labelenumi{\arabic{enumi}.}
\setcounter{enumi}{3}
\tightlist
\item
  There are two boxes on stage. Box B contains 10 blue balls, and the
  other, Box W, contains 6 white balls. A common fair die is rolled by
  the game-master and if the roll of the die is x, then xblue balls are
  removed from Box B and these x blue balls are placed in Box W.
  Finally, after stirring, a ball is drawn at random from Box W.
\end{enumerate}

\subsubsection{{[}a{]} Find the probability that the final ball drawn
from Box W is blue.
{[}10{]}}\label{a-find-the-probability-that-the-final-ball-drawn-from-box-w-is-blue.-10}

\subsubsection{{[}b{]} Find the conditional distribution of the die roll
given that the final ball drawn from Box W is blue.
{[}16{]}}\label{b-find-the-conditional-distribution-of-the-die-roll-given-that-the-final-ball-drawn-from-box-w-is-blue.-16}

\subsubsection{{[}c{]} Compare and contrast the conditional distribution
of die rolls in {[}b{]} with the distribution of the die rolls before we
learned that the ball from Box W was blue.
{[}8{]}}\label{c-compare-and-contrast-the-conditional-distribution-of-die-rolls-in-b-with-the-distribution-of-the-die-rolls-before-we-learned-that-the-ball-from-box-w-was-blue.-8}

\newpage

\subsection{Problem 5}\label{problem-5}

\begin{enumerate}
\def\labelenumi{\arabic{enumi}.}
\setcounter{enumi}{4}
\tightlist
\item
  The probability p that an old component of robotic machinery success-
  fully attaches two engine parts is modeled by the prior distribution
  π(p) = 2p , 0 \textless p\textless1. The variable of interest in a
  factory experiment is X = the number of attempts required until the
  robot successfully attaches the two parts, and you may assume that
  attempts are independent, and each has probably of success p.
\end{enumerate}

\subsubsection{a{]} For starters, show that π(p) is a valid prior
distribution.
{[}6{]}}\label{a-for-starters-show-that-ux3c0p-is-a-valid-prior-distribution.-6}

\subsubsection{b{]} Find the joint distribution for the random variables
p and X.
{[}10{]}}\label{b-find-the-joint-distribution-for-the-random-variables-p-and-x.-10}

\subsubsection{c{]} Find the marginal distribution for the random
variable X. In other words, find a formula for P(X= x) for x=
1,2,3,\ldots.
{[}12{]}}\label{c-find-the-marginal-distribution-for-the-random-variable-x.-in-other-words-find-a-formula-for-px-x-for-x-123.-12}

\subsubsection{d{]} Find the posterior distribution for pgiven that X=
xis observed in the experiment.
{[}10{]}}\label{d-find-the-posterior-distribution-for-pgiven-that-x-xis-observed-in-the-experiment.-10}

\subsubsection{e{]} In Bayesian inference, often the mean (or
expectation) of the posterior distribution is used as an estimate of p
based on the evidence --- for this experiment, find this posterior mean
as a function of the observed data x.
{[}8{]}}\label{e-in-bayesian-inference-often-the-mean-or-expectation-of-the-posterior-distribution-is-used-as-an-estimate-of-p-based-on-the-evidence-for-this-experiment-find-this-posterior-mean-as-a-function-of-the-observed-data-x.-8}

\newpage

\#\#Problem 6

\begin{enumerate}
\def\labelenumi{\arabic{enumi}.}
\setcounter{enumi}{5}
\tightlist
\item
  Let Ω be the region defined by: Ω = \{(x,y) : 0 \textless x\textless2,
  0 \textless y\textless{} x/2\} I recommend you graph the region Ω.
  Suppose (X,Y) are jointly continuous random variables distributed on Ω
  with joint density
\end{enumerate}

f(x,y) = (1/3)(y+ 2x), (x,y) ∈ Ω,

and f(x,y) = 0 otherwise.

\subsubsection{{[}a{]} Verify that f(x,y) is a valid density function.
{[}10{]}}\label{a-verify-that-fxy-is-a-valid-density-function.-10}

\subsubsection{{[}b{]} Find the marginal density for X.
{[}8{]}}\label{b-find-the-marginal-density-for-x.-8}

\subsubsection{{[}c{]} Find the marginal density for Y.
{[}10{]}}\label{c-find-the-marginal-density-for-y.-10}

\subsubsection{{[}d{]} Find the conditional density of Y given X.
{[}8{]}}\label{d-find-the-conditional-density-of-y-given-x.-8}

\subsubsection{{[}e{]} Find the expectation of Y given that X= x.
{[}10{]}}\label{e-find-the-expectation-of-y-given-that-x-x.-10}

\subsubsection{{[}f{]} If we know that X = 1, find the probability that
Y is in the interval (1/4, 1/2).
{[}10{]}}\label{f-if-we-know-that-x-1-find-the-probability-that-y-is-in-the-interval-14-12.-10}

\end{document}
